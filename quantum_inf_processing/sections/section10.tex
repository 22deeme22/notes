\section{Quantum magnetic moment}
    The energy of the interaction of classical magnetic moment with magnetic field is $\propto - \vec{B}\cdot  \vec{M}$.
    But electrons, protons, etc\ldots posses also intrensic $\vec{M}$ (magnetic moment) with a \important{quantum observable}. Let's see the quantum magnetic moment of type spin$\frac{1}{2}$. 
    \[\text{Energy observable } H = -\gamma \frac{\hbar}{2}\vec{B}\cdot \vec{\sigma}, \mathspace \text{  where } \vec{\sigma}=\left(\sigma_x, \sigma_y, \sigma_z\right), \mathspace \hbar = \frac{h}{2\pi} \text{ (Planck constant)}.\]
    \begin{remark}{Remark}
        \begin{tcolorbox}[gris]
            Note that for 
            \begin{itemize}[left=10pt, label=\textbullet]
                \item spin$\frac{1}{2}$: $\sigma_z = \begin{pmatrix} 1 & 0 \\ 0 & -1 \end{pmatrix} , \mathspace \sigma_x =\begin{pmatrix} 0 & 1 \\ 1 & 0 \end{pmatrix} , \mathspace \sigma_y = \begin{pmatrix} 0 & -i \\ i & 0 \end{pmatrix} $, 
                \item spin: matrices are 3 x 3, 
                \item spin$\frac{3}{2}$: matrices are 4 x 4,
                \item in general: spin s: matrices are $\left(2s+1\right)$ x $\left(2s+1\right)$.
            \end{itemize}
        \end{tcolorbox}
    \end{remark}
    \[H = -\gamma\frac{\hbar}{2}\vec{B}\cdot \vec{\sigma}= -\gamma \frac{\hbar}{2}\left(B_x\sigma_x+B_y\sigma_y+B_z\sigma_z\right)=-\gamma\frac{\hbar}{2}\begin{pmatrix} B_z & B_x-iB_y \\ B_x+iB_y & -B_z \end{pmatrix} ,\]
    this is a Hermitian matrix, $H^{\dagger}=H \to $ the spectrum of eigenvalues is real, eigenvectors form an orthogonal basis of $\mathbb{C}^2$. Without loss of genrality, we have the same physical situation for 
    \[\text{with } \gamma B = \omega \text{ (unit of frequency)}, \mathspace \mathspace\vec{B}=\left(0,0,B\right) \implies H=-\gamma \frac{\hbar}{2}B\cdot \sigma_z= \frac{\hbar \omega}{2}\begin{pmatrix} 1 & 0 \\ 0 & -1 \end{pmatrix} \implies \text{ diagonal !} \implies\]
    \begin{center}
        \begin{tikzpicture}[scale=1.2, every node/.style={font=\small}]
            \draw[->, black]
                (0,-2.0) -- (0,1);
            \draw (-0.05,0.4) -- (0.05, 0.4)
            \draw (-0.05,-1.4) -- (0.05, -1.4)
            \node at (-0.4, 0.4) {$+\frac{\hbar \omega}{2}$}
            \node at (1.2, 0.4) {$\ket{\downarrow}$ Excited state}

            \node at (1.2, -1.4) {$\ket{\uparrow}$ Ground state}
            \node at (-0.4, -1.4) {$-\frac{\hbar \omega}{2}$}
        \end{tikzpicture}
    \end{center}
    In general $\omega = \gamma \sqrt{B_x^2+B_y^2+B_z^2} \implies$ energy depends only of norm of $\vec{B}$.

    \noindent Eigenvectors: $\begin{pmatrix} 1 \\ 0 \end{pmatrix} = \ket{0} =\ket{\uparrow}, \mathspace \begin{pmatrix} 0 \\ 1 \end{pmatrix} =\ket{1}=\ket{\downarrow}$.
    \subsection{Time evolution}
        If we only have $\vec{B}=\left(0,0,B_0\right), \mathspace B_0>0$,
        \begin{enumerate}[left=10pt]
            \item Compute time evolution: $U\left(t\right)=\ket{\psi_0}=\ket{\psi_t}$,\\
            \item Schrodinger equation: 
            \begin{align*} 
                i\hbar\frac{d}{dt}\ket{\psi_t}=H\ket{\psi_t} &\iff i\hbar \frac{d}{dt} U_t = HU_t, \\
                U_t = \exp\left(\frac{it}{\hbar}H\right)=\exp\left(\frac{it}{\hbar}\frac{\hbar \omega_0 \sigma_z}{2}\right) &\implies U_t = \exp \left[\frac{-it \omega_0}{2}\begin{pmatrix} 1 & 0 \\ 0 & -1 \end{pmatrix}\right] = \begin{pmatrix} e^{\frac{-it\omega_0}{2}} & 0 \\ 0 & e^{\frac{it\omega_0}{2}}  \end{pmatrix}, \\
                \ket{\psi_0} = \left(\cos\frac{\theta}{2}\right)\ket{0}+e^{i\phi}\left(\sin\frac{\theta}{2}\right)\ket{1} &\implies \ket{\psi_t}=U_t\ket{\psi_0} = e^{\frac{-i^\omega_0}{2}} \left[\left(\cos\frac{\theta}{2}\right)\ket{0}+e^{i\left(\phi-\omega_0t\right)}\left(\sin\frac{\theta}{2}\ket{1}\right)\right].
            \end{align*}
        \end{enumerate}
        But now, if we have a rotating magnetic field in (xy) plane: $\vec{B}\left(t\right)=\left(B_1\cos\omega t, B_1\sin\omega t, B_0\right)$, the time evolution of $\ket{\psi_t} \in \mathbb{C}^2$ generated by Hamiltonian: 
        \[H = -\frac{\gamma \hbar}{2}\vec{B}\left(t\right)\cdot \vec{\sigma}, \mathspace \mathspace \vec{\sigma}=\left(\sigma_x,\sigma_y,\sigma_z\right), \mathspace\mathspace \ket{\psi_t}=U_t\ket{\psi}.\]
        To find $U_t$, we use Schrodinger equation
        \begin{enumerate}[left=10pt]
            \item For the constant case $\vec{B}=\left(0,0,B_0\right)$, the solution is time independant $\implies$ is a constant. 
            \begin{align*} 
                &i \hbar \frac{d}{dt}U_t =H\left(t\right)U_t, \\
                &H\left(t\right)=-\gamma\frac{\hbar}{2}B_0\sigma_z, \text{ (H(t) is time independant!)} \\
                &U_t=\exp\left(-\frac{it}{\hbar}H\right).
            \end{align*}
            
            \item For the rotating (xy) plan $\vec{B}=\left(B_1\cos\omega t, B_1\sin\omega t, B_0\right)$, we have 
            \[i\hbar\frac{d}{dt}U_t=H\left(t\right)U_t=-\gamma\frac{\hbar}{2}\vec{B}\left(t\right)\cdot \vec{\sigma}U_t = \left[-\frac{\hbar \omega_1}{2}\left(\left(\cos\omega t\right)\sigma_x +\left(\sin\omega t\right)\sigma_y\right)-\frac{\hbar\omega_0}{2}\sigma_z\right]U_t.\] 
            \[\text{With } \gamma B_1=\omega_1, \gamma B_0=\omega_0.\]
        \end{enumerate}
    \subsection{Change of reference frame}
        We are going to a rotating frame to have an independant $H\left(t\right)$. 
        
        \noindent State in rotating frame: $\ket{\widetilde{\psi}}=e^{-i\frac{\omega t}{2}\sigma_z}$,

        \noindent State in lab frame: $\ket{\psi_t}=e^{\frac{it}{\hbar}K}\ket{\psi_t}$,     
        
        \noindent New hamiltonian in rotating frame: $\ket{\widetilde{\psi}_t}=\underbrace{e^{i\frac{t}{\hbar}K}U_t}_{\widetilde{U}_t}\underbrace{\ket{\psi_0}}_{\ket{\widetilde{\psi}_0}}$. 
        \begin{align*} 
            i\hbar\frac{d}{dt}\widetilde{U}_t=i\hbar \frac{d}{dt}\left(e^{i\frac{t}{\hbar}K}U_t\right)=-Ke^{i\frac{t}{\hbar}K}U_t+e^{i\frac{t}{\hbar}K}\underbrace{\left(i\hbar\frac{d}{dt}U_t\right)}_{H\left(t\right)U_t} = \\
            \left[-Ke^{it\frac{K}{\hbar}}+e^{it\frac{K}{\hbar}}H\left(t\right)\right]=\left[-K+e^{it\frac{K}{\hbar}}H\left(t\right)e^{-it\frac{K}{\hbar}}\right]\underbrace{e^{it\frac{K}{\hbar}}U_t}_{\widetilde{U}_t}.
        \end{align*}
        \begin{tcolorbox}[bleu]
        Then, in rotating frame: 
        \[i\hbar \frac{d}{dt}\widetilde{U}_t=\widetilde{H}\left(t\right)\widetilde{U}_t, \mathspace \mathspace \widetilde{H}\left(t\right)=\left[-K+e^{it\frac{K}{\hbar}}H\left(t\right)e^{-it\frac{K}{\hbar}}\right],\]
        \[K = -\hbar\frac{\omega_0}{2}\sigma_z=i\frac{\omega_0}{2}\hbar \begin{pmatrix} 1 & 0 \\ 0 & -1 \end{pmatrix}, \mathspace \mathspace e^{it\frac{K}{\hbar}}= \begin{pmatrix} e^{-i\frac{\omega_0t}{2}} & 0 \\ 0 & e^{i\frac{\omega_0t}{2}} \end{pmatrix} .\]
        \end{tcolorbox}
        \begin{remark}{Remark}
           \begin{tcolorbox}[gris]
                \[\widetilde{H}\left(t\right)=\frac{\hbar}{2}\delta\sigma_z-\frac{\hbar\omega}{2}\sigma_x, \mathspace \mathspace \delta=\omega-\omega_0 \leftarrow \text{ time independant!}\]
           \end{tcolorbox}
        \end{remark}
        Let's compute evolution matrix in rotating frame: 
        $\widetilde{U}_t=\exp\left(-\frac{it}{2}\left[\delta\sigma_z-\omega_1\sigma_x\right]\right)$.

        \noindent Recall generalized Euler (type) formula $e^{i\alpha\hat{m}\cdot \vec{\sigma}} =\left(\cos\alpha\right)I+i\left(\sin\alpha\right)\hat{m}\cdot \vec{\sigma}$. with 
        \[\alpha= -\frac{t}{2}\sqrt{\delta ^2 +\omega_1^2}, \mathspace \mathspace m_z=\frac{\delta}{\sqrt{\delta ^2 +\omega_1^2}}, \mathspace \mathspace m_x=\frac{-\omega_1}{\sqrt{\delta ^2+\omega_1^2}}, \mathspace \mathspace m_y = 0,\]
        we find 
        \[U_t= \begin{pmatrix} u_{\uparrow \uparrow}\left(t\right) & u_{\uparrow \downarrow}\left(t\right) \\ u_{\downarrow \uparrow}\left(t\right) & u_{\downarrow \downarrow}\left(t\right) \end{pmatrix} .\]
    To go back to the lab frame: $\widetilde{U}_t: e^{i\frac{t}{\hbar}K}U_t \implies U_t=e^{-i\frac{t}{\hbar}K}\widetilde{U}_t=\begin{pmatrix} u_{\uparrow \uparrow}\left(t\right) & u_{\uparrow \downarrow}\left(t\right) \\ u_{\downarrow \uparrow}\left(t\right) & u_{\downarrow \downarrow}\left(t\right) \end{pmatrix}$ .

    \noindent We can look at some probabilitie: 
    \begin{align*} 
        &u_{\downarrow \uparrow}=\bra{\downarrow}U\ket{\uparrow}, \mathspace \mathspace \ket{\uparrow}=\ket{0}, \mathspace \mathspace \ket{\downarrow}=\ket{1}, \\
        &\left|u_{\uparrow \downarrow}\right|^2=\left|\bra{\downarrow}U_t\ket{\uparrow}\right|^2=prob\left(\ket{\uparrow}\to\ket{\downarrow}\right), \\
        &\left|u_{\downarrow \uparrow}\right|^2 = \ldots  \\
        prob\left(\ket{\uparrow} \text{ initially} &\to \frac{\ket{\uparrow}+\ket{\downarrow}}{\sqrt{2}} \text{ at time t}\right) =\left|\left(\frac{\ket{\uparrow}+\ket{\downarrow}}{\sqrt{2}}\right)U_t\ket{\downarrow}\right|^2=\frac{1}{2}\left|u_{\uparrow \uparrow}\left(t\right)+u_{\downarrow \uparrow}\left(t\right)\right|^2.
    \end{align*}
 
