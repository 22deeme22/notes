\section{Applications of principles}
    \subsection{Mach-Zehnder interferometer}
       \includegraphics[width=5cm]{images/1.png}
       \begin{itemize}[left=10pt, label=\textbullet]
           \item \textbf{Space:} $\mathcal{H}=\mathbb{C}^2=\{\begin{pmatrix} \alpha \\ \beta \end{pmatrix}, \mathspace \alpha & \beta \in \mathbb{C}, \left|\alpha\right|^2+\left|\beta\right|^2=1\}$.
           \item \textbf{Basis:} $\begin{pmatrix} 1 \\ 0 \end{pmatrix} = \ket{H}, \mathspace \begin{pmatrix} 0 \\ 1 \end{pmatrix} = \ket{V}, \mathspace \alpha\ket{H}+\beta\ket{V}=\ket{\psi}$.
           \item \textbf{Beam splitter:} $U = \frac{1}{\sqrt{2}}\begin{pmatrix} 1 & 1 \\ 1 & -1 \end{pmatrix} \left(\text{ Hadamard matrix}\right)$.
           \item \textbf{State after the beam splitter: }$U\ket{H}= \frac{1}{\sqrt{2}}\left(\ket{H}+\ket{V}\right)$. \begin{remark}{Remark}
                   \begin{tcolorbox}[gris]
                        It's a very srange state that is at the same time horizontal and vertical. If it wasn't a qubit, it would either be completely reflected or it would go through. 
                   \end{tcolorbox}
           \end{remark}
       \item \textbf{Perfect mirror: } $R = \begin{pmatrix} 0 & 1 \\ 1 & 0 \end{pmatrix}, \mathspace RR^{\dagger}=R^{\dagger}R=1 \left(\text{\important{important condition!}}\right), \mathspace R^{\dagger}=R \left(\text{coincidence}\right)$.
       \item \textbf{State after the perfect mirror: }$RU\ket{H}=\frac{1}{\sqrt{2}}\left(R\ket{H}+R\ket{V}\right)=\frac{1}{\sqrt{2}}\left(\ket{V}+\ket{H}\right)$.
       \item \textbf{State after the second beam splitter: }$URU\ket{H}=\frac{1}{\sqrt{2}}\left(U\ket{V}+U\ket{H}\right)$.
       \item \textbf{State before the detector: }$\psi_{\text{before detector}}=2\cdot \frac{1}{2}\ket{H}=\ket{H}$.
       \item \textbf{Measurement: } model with orthogonal basis of $\mathcal{H}=\mathbb{C}^2= \{\ket{H} \text{and} \ket{V}\}$:

           At the end of the interferometer, we measure state $\ket{H} \text{ or state } \ket{V}$, 

           if we measure $\ket{H} \implies$ clic in $D_1 \implies$ register +1,

           if we measure $\ket{V}\implies$ clic in $D_1 \implies$ register -1.

           By the Born rule : 
           \begin{align*} 
               prob\left(+1\right)=\left|\braket{H|\psi_{\text{before detector}}}\right|^2 &= 1 \\
               prob\left(-1\right)=\left|\braket{V|\psi_{\text{before detector}}}\right|^2 &= 0
           \end{align*}
           If we follow the \textit{\hyperlink{princ4}{$4^{\text{th}}$ principle}}, then we know that $\ket{H}$ and $\ket{V}$ are the eigenvectors of our observable and $\left(+1\right)$ and $\left(-1\right)$ are the eigenvalues of our observable.

           By spectral theorem, we define our observable: $\left(+1\right)\ket{H}\bra{H}+\left(-1\right)\ket{V}\bra{V}=\begin{pmatrix} +1 & 0 \\ 0 & -1 \end{pmatrix} = Z$.

           Expectated value of (Z): $\bra{\psi_{\text{before detector}}}Z\ket{\psi_{\text{before detector}}}=\bra{H}\{\left(+1\right)\ket{H}\bra{H}+\left(-1\right)\ket{V}\bra{V}\}\ket{H}=\left(+1\right)\braket{H|H}\braket{H|H}+\left(-1\right)\braket{H|V}\braket{H|V}=\left(+1\right)$.

           
           Var(Z): $\bra{\psi}Z^2\ket{\psi}-\bra{\psi}Z\ket{\psi}^2=0$.
       \end{itemize}
    \subsection{Photon polarization}
        \begin{itemize}[left=10pt, label=\textbullet]
            \item \textbf{Classical electro-magnetic waves: }
                \begin{align*}
                    \text{Linear polarization: } \vec{E} &\propto \vec{E}_0 \begin{pmatrix} \cos \theta \\ \sin \theta \\ 0 \end{pmatrix} e^{i \omega t} e^{i\frac{2\pi}{\lambda}z},\\
                    \text{Circular polarization: } \vec{E} &\propto \vec{E}_0 \begin{pmatrix} 1 \\ i \\ 0 \end{pmatrix} e^{i\omega t} e^{i\frac{2\pi}{\lambda}z}.
                \end{align*}
                \begin{remark}{Remark}
                    \begin{tcolorbox}[gris]
                       The polarization of an electro-magnetic wave indicate how the electrique field oscillate in space and time (it shows the direction).
                    \end{tcolorbox}
                \end{remark}
            \item \textbf{Photon quantum particles: }Photon quantum particles that carry electro-magnetic energy have a polarization state:  
                
                $\text{Linear polarization state: } \ket{\theta}&=\cos \theta \ket{x}+\sin \theta \ket{y}, \mathspace
                 \ket{\theta_{\perp}} &= -\sin \theta \ket{x} + \cos \theta \ket{y}.$

                 Circular polarization state: $\ket{\circlearrowleft} = \frac{1}{\sqrt{2}}\left(\ket{x}+i\ket{y}\right)=\frac{1}{\sqrt{2}}\begin{pmatrix} 1 \\ i \end{pmatrix}, \mathspace \ket{\circlearrowright}= \frac{1}{\sqrt{2}}\left(\ket{x}-i\ket{y}\right)=\frac{1}{\sqrt{2}}\begin{pmatrix} 1 \\ -i \end{pmatrix}$.
            \begin{remark}{Remark}
                \begin{tcolorbox}[gris]
                    A classical electro-magnetic wave can be viewed as a collective state of many photons, each having the same polarization state. This photon polarization state determine the direction and the form of oscillation of the electric field in the corresponding classical wave.
                \end{tcolorbox}
                
            \end{remark}
            
        \end{itemize}

