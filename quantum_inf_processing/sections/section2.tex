\section{5 principles of quantum mechanics}
    \subsection{State of a system}
        \begin{tcolorbox}[vert]
            The state of an \important{isolated} system is a vector $\ket{\phi} \in \mathcal{H}$ in Hilbert space $\mathcal{H}$ w.r.t the normalization condition $\braket{\phi|\phi}=1$.       
        \end{tcolorbox}
        \begin{remark}{Remark}
            \begin{tcolorbox}[gris]
                $\ket{\phi}$ and $e^{i\lambda}\ket{\phi}, \lambda \in \mathbb{R}$ are physically equivalent.
            \end{tcolorbox}
        \end{remark}
        \begin{remark}{Examples}
           \begin{tcolorbox}[rouge]
              \begin{enumerate}[left=10pt]
                  \item  Let $\mathcal{H}=\mathbb{C}^2=\{\begin{pmatrix} \alpha \\ \beta \end{pmatrix} | \mathspace\alpha,\beta \in \mathbb{C}\}$ (qubit space), then
                  $\mathbb{C} \implies \ket{\psi}=\alpha\ket{0}+\beta\ket{1}, \mathspace \left|\alpha\right|^2+\left|\beta\right|^2=1=\alpha\alpha^*+\beta\beta^*$,
                  \item Physical system: Mach-Zehnder interferometer 
                  \item Let $\mathcal{H}=\mathbb{C}^d, \mathspace \begin{pmatrix} \alpha_1 \\ \ldots \\ \alpha_d \end{pmatrix} = \ket{\psi}=\sum_{ i = 1 } ^{ d }\alpha_i \ket{i}, \mathspace \sum_{ i=1 } ^{ d }\left(\alpha_i\right)^2=1,$ (qudit space).
              \end{enumerate}
               
           \end{tcolorbox}
        \end{remark}
    \subsection{States evolve with time}
        \begin{tcolorbox}[vert]
           As follows: $\left|\psi_t\right|= U_t\ket{\psi_0}$ where $U_t$ is unitary matrix.
        \end{tcolorbox}
        \begin{remark}{Example}
            \begin{tcolorbox}[rouge]
                We can desribe the transformation of the state of a particle by a prefecting reflecting mirror ($\ket{H} \to \ket{V} \text{ and } \ket{V}\to\ket{H}$), by 
                \[U = \begin{pmatrix} 0 & 1 \\ 1 & 0 \end{pmatrix}, \mathspace \ket{H}=\begin{pmatrix} 1 \\ 0 \end{pmatrix}, \mathspace \ket{V}=\begin{pmatrix} 0 \\ 1 \end{pmatrix},\] 
                \[\implies U\left(\alpha\ket{H} + \beta\ket{V}\right)= \beta\ket{H}+\alpha\ket{V}.\]
            \end{tcolorbox}
        \end{remark}
    \subsection{Observable quantities}
        \begin{tcolorbox}[vert]
            Quantities that we measure: ``observables'' are given by Hermitian matrices of dimension $\mathcal{H} \cdot \mathcal{H}$
        \end{tcolorbox}
        \begin{remark}{Example}
            \begin{tcolorbox}[rouge]
                Observable for 1 qubit $\left(\mathbb{C}^2\right)$: 
                \[A=a\mathbbm{1}+bX+cY+dZ, \mathspace A = a^{\dagger} \implies a,b,c,d \in \mathbb{R}\]
                \[\mathbbm{1}=\begin{pmatrix} 1 & 0 \\ 0 & 1 \end{pmatrix}, X= \begin{pmatrix} 0 & 1 \\ 1 & 0 \end{pmatrix}, Y=\begin{pmatrix} 0 & -i \\ i & 0 \end{pmatrix}, Z = \begin{pmatrix} 1 & 0 \\ 0 & -i \end{pmatrix}.\]
            \end{tcolorbox}
        \end{remark}
        \hypertarget{princ4}{}
        \subsection{Measurement give probability outcome}
            \begin{tcolorbox}[vert]
                When observable A is measured, the outcome is a random eigenvalue of A, call it $\lambda_i, i = 1, \ldots, d = dim \mathcal{H}$, the original state before measure $\ket{\psi}$ becomes after measurement an output state $\ket{v_i}$ with eigenvalues corresponding to $\lambda_i$. The probability distributioin associated to the output $\lambda_i, \ket{v_i}$ is $prob\left(i\right)=\left|\braket{v_i|\psi}\right|^2 \leftarrow $ \textbf{BORN RULE}.
            \end{tcolorbox}
            \begin{itemize}[left=10pt, label=\textbullet]
                \item Results of the measurement are eigenvalues $\lambda_i$ of observable A, 
                \item Every eigenvalue is associated to an eigenvector $\ket{v_i}$, 
                \item If the system is in the original state $\ket{\psi}$, the probability to obtain $\lambda_i$ is \[P\left(\lambda_i\right)=\left|\braket{v_i|\psi}\right|^2,\]
                \item After the measurement, the state collapse on the eigenvector: $\ket{\psi} \to \ket{v_i}$.
            \end{itemize}
            \begin{theoreme}
                Spectral Theorem: 

                Let A be an hermitian matrice $A = A^{\dagger}$ and let $A\ket{v_i}= \lambda_i\ket{v_i}, i = 1, \ldots, d = dim\mathcal{H}$
                \begin{itemize}[left=10pt, label=\textbullet]
                    \item $\lambda_i \in \mathbb{R}$,
                    \item $\ket{v_1}, \ldots, \ket{v_d}$ form an orthogonal basis,
                \end{itemize}
                    \important{$ \implies A=\sum_{ i =1 } ^{ d }\lambda_i \ket{v_i}\bra{v_i}= \begin{pmatrix} \lambda_1 & \ldots & 0 \\  & \ldots &  \\ 0 & \ldots & \lambda_d \end{pmatrix} $}.
            \end{theoreme}
            \begin{itemize}[left=10pt, label=\textbullet]
                \item Lemma: $\sum_{ i=1 } ^{ d }\left|\braket{v_i|\psi}\right|^2=1$,  because $\left\|\psi\right\|=1$.
                \item Property: 
                    \begin{itemize}[left=10pt, label=\textendash]
                        \item $E\left(A\right)=\sum_{ i=1 } ^{ d }\lambda_i \left|\braket{v_i|\psi}\right|^2=\bra{\psi}A\ket{\psi}$
                        \item Var$\left(A\right)=\bra{\psi}A^2\ket{\psi}-\bra{\psi}A\ket{\psi}^2$
                    \end{itemize}
            \end{itemize}
        \subsection{Composite system and entanglement}
            \begin{tcolorbox}[vert]
                The composite system of $\mathcal{H}_A$ and $\mathcal{H}_B$ is equal to $\mathcal{H}_{A\cup B}= \mathcal{H}_A \otimes \mathcal{H}_B$.
            \end{tcolorbox}
            \begin{remark}{Remark}
                \begin{tcolorbox}[gris]
                    $dim\left(\mathcal{H}_A \otimes \mathcal{H}_B\right)=\left(dim\mathcal{H}_A\right)\left(dim\mathcal{H}_B\right)$.
                \end{tcolorbox}
            \end{remark}
            \begin{remark}{Example}
                \begin{tcolorbox}[rouge]
                    $\mathcal{H}_A = \mathbb{C}^2, \mathspace \mathcal{H}_B=\mathbb{C}^2, \mathspace \begin{functionbypart}{\mathbb{C}^2 \otimes \mathbb{C}^2}
                        \ket{00} = \ket{0} \otimes \ket{0}  \\
                        \ket{01} = \ket{0} \otimes \ket{1}  \\
                        \ket{10} = \ket{1} \otimes \ket{0}  \\
                        \ket{11} = \ket{1} \otimes \ket{1}  
                    \end{functionbypart}
                    $ 
                \end{tcolorbox}
            \end{remark}
            \textbf{Product states:} $\ket{\psi}=\ket{\phi_A} \otimes \ket{\chi_B}, \mathspace \psi \in \mathcal{H}_A \otimes \mathcal{H}_B,  \mathspace \phi \in \mathcal{H}_A, \mathspace \chi \in \mathcal{H}_B$.
            \begin{tcolorbox}[vert]
                Entangled states: $\nexists$ a factorisation of the state.
            \end{tcolorbox}
            \begin{remark}{Example}
                \begin{tcolorbox}[rouge]
                    
                    $\ket{\psi}=\frac{1}{\sqrt{2}}\left(\ket{0}\otimes\ket{0}+\ket{1}\otimes \ket{1}\right) \neq \left(\alpha\ket{0}+\beta\ket{1}\right)\otimes\left(\gamma\ket{0}+\delta\ket{1}\right)$.
                \end{tcolorbox}
            \end{remark}
            \begin{remark}{Apparté}
                \textbf{Bloch sphere of a qubit state vector:}

                $\ket{\psi}=\alpha\ket{0}+\beta\ket{1} = \cos\left(\frac{\theta}{2}\right)\ket{0} + e^{i\phi}\sin\left(\frac{\theta}{2}\right)\ket{1}$.
            \end{remark}
